\documentclass[a4paper, 12pt]{article}

\usepackage[brazilian]{babel}
\usepackage[utf8]{inputenc}
\usepackage[T1]{fontenc}
\usepackage{verbatim}
\usepackage{hyperref}

\title{Manual AMPL}
\author{Viviane J Galvão}
\date{\today}

\begin{document}
\maketitle

\section{Instalação}
A instalação do pacote gratuito do AMPL (versão para estudantes) pode ser realizada através do link: \url{http://ampl.com/try-ampl/download-a-free-demo/}.\\

Essa é uma versão grátis, a qual permite a resolução de problemas com até 500 variáveis, com 500 restrições lineares ou com 300 restrições não-lineares. Ainda, essa versão acompanha os principais resolvedores disponíveis para AMPL.\\

Os principais arquivos para o uso do ambiente AMPL são o executável \verb|ampl| e a sua licensa \verb|ampl.lic|, os quais podem ser movidos diretamente para o diretório onde se encontra o resolvedor a ser utilizado.\\

Após, para que seja possível incluir o nosso resolvedor em C\verb|++| ao AMPL, é preciso baixar o diretório com todos os arquivos disponíveis em: 
\begin{center}
\url{http://ampl.com/netlib/ampl/solvers/index.html}
\end{center}
Esse diretório contém a fonte para a biblioteca de rotinas que auxiliam resolvedores a funcionarem com AMPL. Essa biblioteca é chamada de \verb|amplsolver.a|, para o caso de sistemas Unix. Os serviços providos por essa biblioteca incluem a leitura de arquivos de saída \verb|.nl| do AMPL e escrita de arquivos de solução \verb|.sol|. Para compilar o \verb|amplsolver.a|, é preciso, via terminal, executar os seguintes comandos:
\begin{enumerate}
\item No diretório baixado, executar
\begin{verbatim}
./configure
\end{verbatim}
para criar o diretório \verb|sys.'uname -m'.'uname -s'|, específico para o sistema operacional utilizado. 
\item Execute:
\begin{verbatim}
cd sys.`uname -m'.`uname -s'
make
\end{verbatim}
e a biblioteca \verb|amplsolver.a| será criada nesse diretório específico para o sistema.
\end{enumerate} 

\section{Resolvendo Problemas em AMPL}
Para códigos em C\verb|++|, é necessário incluir o arquivo \verb|asl.h|, o qual facilita a leitura e acesso às informações dos problema a ser resolvido, como número de variáveis, restrições e objetivos.
\begin{verbatim}
#include `asl.h'
\end{verbatim}

Esse arquivo \verb|asl.h| é encontrado na pasta de \textit{includes} necessários para a conexão do nosso resolvedor e o AMPL. No caso, essa pasta contém todos os arquivos \verb|*.h| e \verb|*.c| contidos do subdiretório \verb|sys.'uname -m'.'uname -s'|. \\

A partir do arquivo $asl.h$ é necessário declarar um ponteiro do tipo \verb|ASL|, o qual será utilizado diretamente para ler e acessar o modelo do problema. 
A função \verb|ASL_alloc|($ASLtype$) aloca um objeto do tipo \verb|ASL| e o parâmetro $ASLtype$ determina quantas não-linearidades são tratadas. No caso, como parâmetro foi utilizado \verb|ASL_read_fg|, o qual auxilia na leitura e tratamento de problemas não-lineares. Mais argumentos e suas características podem ser encontrados em \textbf{[1]}. \\

Inicialmente, o nome do problema no formato \textit{.mod} deve ser informado como parâmetro para o código principal do seu resolvedor (no caso, o próprio ambiente AMPL informa o problema como parâmetro como resolvedor). O nome da instância é armazenado na variável \verb|stub| e é utilizada para acessar o arquivo através da função \verb|jac0dim(char *stub, fint stub_len)|. \\ 

A partir do arquivo \verb|.nl| retornado por \verb|jac0dim(char *stub, fint stub_len)|, é possível extrair todas as informações do problema informado através da função \verb|fg_read_ASL(ASL *asl, FILE *nl, int flag)|, a qual auxilia na leitura e tratamento de problemas não-lineares.\\  

Através da função \verb|objval(int nobj, double *x, fint *nerror)| é possível obter o valor do objetivo \verb|nobj| para um ponto \verb|*x| escolhido. Se o problema possui um objetivo, então estamos interessados em obter o valor do objetivo $\verb|nobj| = 0$. Ainda, o argumento \verb|fint *nerror| controla o que acontece se ocorre algum erro na função de avaliação. \\

Para facilitar a utilização dessa função de avaliação em qualquer parte do código do resolvedor, declara-se uma função na estrutura principal do resolvedor que, no caso mono-objetivo, recebe um valor no tipo \verb|double| e retorna o valor da função de avaliação \verb|objval|. Dessa forma, essa nova função pode ser passada por parâmetro para outras rotinas do nosso resolvedor e a função de avaliação estará sempre acessível sem ser necessário todo o processo de leitura do modelo.\\ 

No caso do problema possuir limites sobre as variáveis do problema, essas informações podem ser acessadas através dos vetores \verb|LUv| e \verb|Uvx|, limites inferiores e superiores, respectivamente. Ainda, quando \verb|Uvx| é nulo, é possível acessar tanto os limites inferiores e superiores através do vetor \verb|LUv|, onde os índices pares armazenam os limites inferiores e os índices ímpares armazenam limites superiores.\\

Ainda, no caso do problema possuir restrições, essas informações podem ser acessadas através dos vetores \verb|LUrhs| e \verb|Urhsx|, as restrições inferiores e superiores, respectivamente. Da mesma forma que no caso de limites sobre as variáveis, se \verb|Urhsx| for nulo, é possível acessar ambas restrições através do vetor \verb|LUrhs|.\\

Uma vez executada a leitura do problema e armazenado todas as informações necessárias, é possível usar essas informações como parâmetros para outras rotinas do nosso resolvedor, assim como a função de avaliação. Na seção \ref{exem1} é possível encontrar um exemplo de como acessar informações de um problema em AMPL não-linear e com limites nas suas variáveis.

\section{Utilizando o ambiente AMPL}
Para sistema operacional Linux, no pacote do AMPL existe o arquivo executável \verb|ampl| e é possível utilizar o ambiente AMPL em terminal executando esse arquivo através de \verb|./ampl|.\\

Uma vez dentro do ambiente AMPL em terminal Linux é possível resolver problemas em arquivos \textit{.mod} utilizando qualquer resolvedor. Para isso, inicialmente é necessário incluir o problema ao ambiente através de:
\begin{verbatim}
ampl: include ack.mod;
ampl: solve;
\end{verbatim}

Quando o comando \verb|solve| é executado, então o problema incluído é resolvido pelo resolvedor padrão que acompanha o AMPL, o MINOS. Para utilizar o nosso resolvedor, é necessário que ele tenha sido compilado a priori e gerado um arquivo executável. Suponha que após compilar no nosso resolvedor tenha sido gerado o executável \verb|solver|. Então, modificamos o resolvedor do AMPL através de:
\begin{verbatim}
ampl: include ack.mod;
ampl: options solver `./solver';
ampl: solve;
\end{verbatim}

Além de permitir que o resolvedor seja modificado, o comando \verb|options| permite modificar outros parâmetros para o ambiente AMPL.\\

Outra alternativa para resolver problemas em AMPL é escrever um \textit{script} de execução em um arquivo \textit{.run}. Uma vez possuindo um \verb|script.run| não é necessário entrar no ambiente \verb|ampl| para incluir os problemas e modificar o resolvedor, basta apenas ter os comandos necessários no arquivo \verb|script.run| e executá-lo através de:
\begin{verbatim}
./ampl script.run
\end{verbatim}

Uma questão importante a ser citada é como compilar o resolvedor de forma a utilizá-lo no ambiente AMPL. Para o C\verb|++|, pode-se proceder da seguinte forma:
\begin{verbatim}
g++ -c -g -Iinclude/ *.cpp -D Param1=$a -D Param2=$b
g++ -o solver main.o Estrategias.o libs/funcadd0.o libs/amplsolver.a -w  
\end{verbatim}

Nesse caso, além do \verb|main.o|, podem ser colocados os nomes de todos os arquivos \verb|*.cpp| que estão incluídos no \verb|main.cpp|. Além, a biblioteca \verb|amplsolver.a| e o arquivo \verb|funcadd0.o| devem também informados no processo de compilação. O arquivo \verb|funcadd0.o| é necessário para quando o modelo \textit{.mod} utiliza funções que são definidas pelo usuário e o resolvedor precisa avaliar essas funções. \\

Note que, se o resolvedor necessita de parâmetros (além do nome do problema, pois esse é informado ao resolvedor de forma automática no ambiente AMPL), é mais fácil informá-los nesse momento, através de \verb|-D Param1=$a| \verb|-D Param2=$b|, onde \verb|Param1| e \verb|Param2| são os nomes parâmetros necessários no resolvedor e \verb|$a| e \verb|$b| são seus valores.

\section{Exemplo de Utilização do AMPL} \label{exem1}
(desenvolvido a partir de exemplos encontrados em \url{http://ampl.com/resources/the-ampl-book/example-files/})
\begin{verbatim}
#include <iostream>
#include <stdio.h>
#include <stdlib.h>
#include "asl.h"
#include "getstub.h"
#include "Estrategias.h"
#include "string.h"
#include <iomanip>

#define INF 1E+21
#define asl cur_ASL

using namespace std;

static fint NERROR = -1;

double objfun(double *x){
    return objval(0, x, &NERROR);
}

int main(int argc, char** argv){
    if(argc > 2){
        NERROR = -1;
        ASL *asl;
        FILE *nl;
        char *stub;

        asl = ASL_alloc(ASL_read_fg);
        stub = argv[1];
        nl = jac0dim(stub, (fint)strlen(stub));

        fg_read_ASL(asl, nl, 0);

        double *lb = NULL;
        double *ub = NULL;

        if(LUv[0] > -Infinity && LUv[1] < Infinity){
            lb = new double[n_var];
            ub = new double[n_var];
            for(int i=0; i<n_var; i++){
                lb[i] = LUv[2*i];
                ub[i] = LUv[2*i+1];
            }
        }
        cout<<";"<<n_var; // numero de variáveis do problema

        double mean = 0;
        for(int seed=1; seed<=SD; seed++){
            mean += PSO(&objfun, n_var, seed, lb, ub);
        }
        mean = mean/(double)SD;

        cout<<";"<<setprecision(10)<<mean;

    }
    else{

    }
    return 0;
}

\end{verbatim}

\section{Exemplo de Script para AMPL} \label{exem2}

\begin{verbatim}
rm resultados.csv
d=31
echo "FUNCAO;DIMENSAO;MEDIA" >> resultados.csv

for f in Instances/*
do
  echo $f
  echo -ne -e $f ";">> resultados.csv

  rm problem.run
  g++ -c -g -Iinclude/ *.cpp -D SD=$d
  g++ -o solver main.o Estrategias.o Solution.o libs/funcadd0.o libs/amplsolver.a -w    
   
  echo "include" $f";" >> problem.run
  echo "options solver './solver';" >> problem.run
  echo "solve;" >> problem.run
  
  ./ampl problem.run >> resultados.csv
  echo >>resultados.csv

done
\end{verbatim} 

\section{Referências}
\textbf{[1]} David M. Gay. Hooking Your Solver to AMPL, 1997.

\end{document}