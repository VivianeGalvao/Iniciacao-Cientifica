\NeedsTeXFormat{LaTeX2e}
%-----------------------------------------------------------
\documentclass[a4paper,12pt]{monografia}
\usepackage[portuguese, colorinlistoftodos, textsize=tiny]{todonotes}
\usepackage{amsmath,amsthm,amsfonts,amssymb}
\usepackage[mathcal]{eucal}
\usepackage{latexsym}
%\usepackage[brazil]{babel}
%\usepackage[latin1]{inputenc}
\usepackage[portuguese]{babel}  
\usepackage[utf8]{inputenc}
\usepackage{setspace}
\usepackage{natbib}
\usepackage{bm}
\usepackage[portuguese,algoruled,longend]{algorithm2e}
\usepackage{listings}
\usepackage{graphicx}
\usepackage{verbatim}

\let \hiddentext \comment %renomeia o comando comment do verbatim como hiddentext
\let \comment \relax %permite que se crie o comando comment do modelo sobrescrevendo o verbatim

\newcounter{todocounter}
\newcommand{\comment}[2][]
{\stepcounter{todocounter}\todo[caption={\thetodocounter: #2}, #1] 
{\begin{spacing}{1}\thetodocounter: #2\end{spacing}}}
\reversemarginpar
\setlength{\marginparwidth}{2.5cm}
\lstloadlanguages{C}
%-----------------------------------------------------------
%-----------------------------------------------------------
\theoremstyle{plain}
\newtheorem{theorem}{Teorema}[section]
\newtheorem{axiom}{Axioma}[section]
\newtheorem{corollary}{Corolário}[section]
\newtheorem{lemma}{Lema}[section]
\newtheorem{proposition}{Proposição}[section]
%-----------------------------------------------------------
\theoremstyle{definition}
\newtheorem{definition}{Definição}[section]
\newtheorem{example}{Exemplo}[section]
%-----------------------------------------------------------
\theoremstyle{remark}
\newtheorem{remark}{Observação}[section]
%-----------------------------------------------------------
%-----------------------------------------------------------
\newcommand{\R}{\mathbb{R}}
\newcommand{\N}{\mathbb{N}}
\newcommand{\Z}{\mathbb{Z}}
\newcommand{\Q}{\mathbb{Q}}
\newcommand{\K}{\mathbb{K}}
\newcommand{\I}{\mathbb{I}}
\newcommand{\id}{\mathbf{1}}
\newcommand{\U}{\mathcal{U}}
\newcommand{\V}{{\cal V}}
%-----------------------------------------------------------
\def\ind{\hbox{ ind }}
%-----------------------------------------------------------
\hyphenation{Con-si-de-ra-mos}
\hyphenation{me-lhor}
\hyphenation{res-pos-ta}
\hyphenation{re-qui-si-tos}




\begin{document}
%%%%%%%%%%%%%%%%%%%%%%%%%%%%%%%%%%%%%%%%%%%%%%%%%%%%%%%%%%%5
%
%                  INFORMAÇÕES PRÉ-TEXTUAIS
%
%----------------- Título e Dados do Autor -----------------
\titulo{Combina\c{c}\~ao de Metaheur\'isticas e Busca Local Determin\'istica e Independente de Derivadas}
%\subtitulo{$<<$Subt\'itulo - opcional$>>$} % opcional
\autor{Viviane de Jesus Galv\~ao} \nome{$<<$Primeiro Nome$>>$} \ultimonome{$<<$\'Ultimo Nome$>>$}
%
%---------- Informe o Curso e Grau -----
\bacharelado %Pode ser \bacharelado \licenciatura \especializacao \mestrado ou \doutorado
\curso{Engenharia Computacional} 
\dia{$<<$dia$>>$} \mes{Junho} \ano{2016} % data da aprovação
\cidade{Juiz de Fora}
%
%----------Informações sobre a Instituição -----------------
\instituicao{Universidade Federal de Juiz de Fora} \sigla{UFJF}
\unidadeacademica{Faculdade de Engenharia}
\departamento{Mec\^anica Aplicada Computacional}
%
%------Nomes do Orientador, 1o. Examinador e 2o. Examinador-
\orientador{Helio Jos\'e Corr\^ea Barbosa}
\ttorientador{$<<$T\'itulo do Orientador$>>$}
%
\coorientador{Heder Soares Bernardino} % opcional
\ttcoorientador{$<<$T\'itulo do Co-orientador$>>$} % se digitado \coorientador
%
\examinadorum{$<<$Nome do Examinador 1$>>$}
\ttexaminadorum{$<<$T\'itulo do Examinador 1$>>$}
%
\examinadordois{$<<$Nome do Examinador 2$>>$}
\ttexaminadordois{$<<$T\'itulo do Examinador 2$>>$}
%
\examinadortres{Nome do Examinador 3}
\ttexaminadortres{T\'itulo do Examinador 3}
%
\examinadorquatro{Nome do Examinador 4}
\ttexaminadorquatro{T\'itulo do Examinador 4}
%
%-------- Informações obtidas na Biblioteca ----------------
%
%\CDU{536.21} \areas{1.Análise Matemática  2. Topologia.}
%\npaginas{xx}  % total de páginas do trabalho
%
%%%%%%%%%%%%%%%%%%%%%%%%%%%%%%%%%%%%%%%%%%%%%%%%%%%%%%%%
%    DADOS DE FORMATACAO DA MONOGRAFIA
%
% A instrução abaixo insere a logo do curso e da instituicao na capa da monografia. Basta comentar caso não queira os logos. Para alterar o logo da instituicao e curso, basta alterar os arquivos logoInstituicao.png e logoCurso.jpg. Caso deseje alterar os arquivos, os substituta por imagens do mesmo tamanho!
\inserirlogo  

\metodologia

\retirarPaginaInicio
%
%
%
%          FIM DAS INFORMAÇÕES PRÉ-TEXTUAIS
%
%%%%%%%%%%%%%%%%%%%%%%%%%%%%%%%%%%%%%%%%%%%%%%%%%%%%%%%%%

\maketitle









%----------------------------dedicatória  opcional--------------
\begin{dedicatoria}
	Aos meus amigos e irmãos.\\
	Aos pais, pelo apoio e sustento.\\
\end{dedicatoria}



%--------Digite aqui o seu resumo em Português--------------
\resumo{Resumo} 
Apesar de serem de fácil compreensão e implementação, os métodos determinísticos independentes de derivadas, como a Busca Padrão, possuem certas limitações devido à sua taxa de convergência lenta e em relação ao quão grande possa ser a dimensão do problema. Já as metaheurísticas, apesar de  sua  robustez,  demandam  demasiado  número  de  avaliações  de  função para  convergirem.

O objetivo deste trabalho é resolver problemas de otimização em espaço de busca contínuo, os quais possuem análise de sua diferenciabilidade inviável e avaliação de alto custo computacional através de métodos de Otimização Sem Uso de Derivadas, além de combinar os métodos de Otimização por Enxame de Partículas e de Evolução Diferencial com a Busca Padrão, avaliar seus resultados e ainda verificar em quais condições este
tipo de combinação gera boas respostas. Um dos principais objetivos deste trabalho é atingir bons resultados reduzindo o número de cálculos da função objetivo. Além disso, propõem-se melhorar a Busca Padrão a fim de diminuir a quantidade de avaliações de função desnecessárias.
\noindent \\ \textbf{Palavras-chave:} Otimização Sem Uso de Derivadas, Metaheurística, Busca Padrão.



%-----------ABSTRACT: Digite aqui o seu resumo em Inglês-------
% Para o Projeto de TCC o Abstract não é utilizado. Na versão final do seu
% TCC basta remover os comentários das próximas duas linhas.

%\resumo{Abstract} You must summarize your work in 150-200 words.

%\noindent \\ \textbf{Keywords:} Monograph, latex, instructions.





%-----------AGRADECIMENTOS: --------------------------------------
% Para o Projeto de TCC os agradecimentos não são utilizado. Na versão 
% final do seu TCC basta remover os comentários das próximas linhas.

%\agradecimento{Agradecimentos} \indent\indent 
%A todos os meus parentes, pelo encorajamento e
%apoio.
%
%Ao professor Beltrano pela orientação, amizade e
%principalmente, pela paciência, sem a qual este trabalho não se
%realizaria.
%
%Aos professores do Departamento de Ciência da Computação pelos seus
%ensinamentos e aos funcionários do curso, que durante esses anos,
%contribuíram de algum modo para o nosso enriquecimento pessoal e
%profissional.
%\newpage


%---------------------- EPÍGRAFE I (OPCIONAL)--------------
\begin{epigrafe}
``Lembra que o sono é sagrado e alimenta de horizontes o tempo acordado de viver''.\\
\hfill Beto Guedes (Amor de Índio)
\end{epigrafe}



%----Sumário, lista de figura e de tabela ------------
 \tableofcontents \thispagestyle{empty} \listoffigures
\thispagestyle{empty} \listoftables \thispagestyle{empty}



%----Glossário ------------
\chapter*{Lista de Abreviações} \addcontentsline{toc}{chapter}{Lista de Abreviações}
\doublespacing  \begin{tabular}{l l}

DCC & Departamento de Ciência da Computação \\
UFJF & Universidade Federal de Juiz de Fora \\

\end{tabular}  \thispagestyle{empty}
%---------------------




%%%%%%%%%%%%%%%%%%%%%%%%%%%%%%%%%%%%%%%%%%%%%%%%%%%%%%%%%%%
%
%--------------Início do Conteúdo---------------------------
%
%
\pagestyle{ruledheader}
\chapter{Introdução}
\section{Apresentação do Tema} %apresentação do tema
No entanto, algumas dificuldades quanto a obtenção dessas informações induzem o uso de uma classe de métodos apropriada:
os métodos de Otimização Sem Derivadas (OSD).

%Pode-se encontrar um denso estudo
%destes métodos em Conn et al. (2009), incluindo análises de convergência e algumas aplicações práticas.
Dentre os métodos da classe OSD, pode-se citar os Métodos de Busca Direta, que utilizam
apenas os valores da função objetivo para convergirem até um ponto de ótimo local. 
%Mais detalhes sobre os Métodos de Busca Direta, como uma abordagem histórica, novas estratégias e um panorama sobre o tratamento de restrições laterais, lineares e não-lineares pode ser encontrado em Kolda et al. (2003).  
Um método de Busca Direta bastante simples é o método de Busca Padrão, que é uma estratégia que opera com movimentos exploratórios, além de ter uma estrutura flexível
que permite sua combinação com outras heurísticas.

Ainda, existem os métodos de natureza estocástica que são populares devido à sua
robustez, mas que possuem alto custo computacional. Estes métodos são conhecidos como metaheurísticas e comumente são inspirados em processos naturais, como é o caso da Otimização por Enxame de Partículas (PSO) o qual é um método populacional inspirado em mecanismos de inteligência coletiva e da Evolução Diferencial (ED) que é inspirado na processo natural de evolução das espécies.

Apesar da sua aplicabilidade, os métodos de otimização sem derivadas têm taxa de convergência baixa e não atingem bons resultados para altas dimensões, diferentemente
dos métodos baseados em derivadas. Pode-se esperar o sucesso desses métodos em problemas
que não tenham mais do que uma centena de variáveis, que possuam custo muito alto de avaliação e nos quais uma rápida taxa de convergência não é o objetivo primordial. 
%Alguns trabalhos
%tentam melhorar a eficiência e robustez desses métodos, 
%como no trabalho de Custódio (2007)que aplica 
%seja aplicando a teoria dos gradientes simpléticos no método de busca padrão 
%e em Vaz and Vicente (2007) que 
%ou combinando a técnica de otimização por enxame de partículas com a busca padrão.

\section{Descrição do Problema}%problema
Frequentemente,  nas Ciências Exatas e nas Engenharias,  são encontrados problemas de
otimização em espaço contínuo com restrições em que o uso de derivadas é inviável seja
pela instabilidade no seu cálculo ou pela inacessibilidade à informações, como em problemas conhecidos
como “caixa preta”,  onde o cálculo da função objetivo e/ou das restrições envolve o uso de
um  simulador.   Além  disso,  esta  classe  de  problemas  frequentemente exige avaliações  com
alto custo computacional. Tais características dificultam a utilização de uma aproximação numérica
das derivadas por diferenças finitas e a de métodos baseados nas técnicas de Newton e quasi-
Newton por envolver alto custo computacional e possível ruído contido na função objetivo ou
restrições do problema. Por exemplo, pode-se citar o problema de projeto de pás do rotor de helicópteros onde o objetivo poderia ser
minimizar a vibração transmitida ao seu ponto central. Neste exemplo, a função objetivo é uma simulação que inclui dinâmica das estruturas e aerodinâmica e cada avaliação de função pode requerer minutos ou até mesmo dias em tempo de CPU.

Além da intenção de resolver o problema de otimização, é necessário obter o máximo de eficiência na busca, encontrando a melhor solução possível demandando o menor número de avaliações de função.

\section{Motivação do Trabalho} %justificativa/motivação
Desde a década de 1960, a demanda de problemas complexos e com pouca informação disponível motivou o desenvolvimento dos métodos
da classe OSD que, até os anos 90, não tinham muita teoria relacionada à garantia de convergência. Atualmente, a ênfase tem sido no entendimento dos métodos já existentes com interesse na sua convergência global. O interesse nesses métodos tem crescido em razão da facilidade de implementação e paralelização, e grande aplicabilidade.

A maioria dos métodos OSD são simples. As principais motivações para a utilização dos métodos OSD estão nos casos onde os problemas possuem avaliações com alto custo computacional e ruido, e não se tem acesso às derivadas e nem pode-se aproximá-las por diferenças finitas e nos casos onde o problema não possui código fonte disponível, o que faz a diferenciação automática ser impossível. Esses problemas são bastante comuns em aplicações em Engenharia e Ciencias Exatas, e geralmente exigem complexas simulações.

A combinação da Busca Padrão com metaherística é uma alternativa para se resolver problemas de OSD com eficiência. As metaheuríscas são capazes de explorar o espaço de busca e escapar de mínimos/máximos locais e cada ótimo global conhecido pela metaheurísca pode ser refinado pela Busca Padrão, através da avaliação da sua vizinhança de acordo um padrão de direções, melhorando a solução da metaheurística.

\section{Hipóteses} %hipoteses
\section{Objetivos} %objetivos
A proposta deste trabalho consiste em combinar uma Busca Padrão com PSO e com ED para resolver
problemas de otimização em espaço de busca contínuo com restrições laterais a fim de melhorar o desempenho das metaheurísticas. Além disso, pretende-se modificar a busca padrão inicialmente adotada com o intuito de realizar menos cálculos da função objetivo, bem como inserir uma estratégia de população dinâmica no PSO que
depende da quantidade de acessos à Busca Padrão. Experimentos computacionais com
problemas geralmente usados na literatura foram realizados para analisar a técnica proposta.


\section{Cronograma}

Para alcançar os objetivos descritos neste projeto, considerando a metodologia apresentada, deve ser desenvolvido o seguinte conjunto de atividades:

\begin{enumerate}
\item Revisão Bibliográfica;
\item Definição da estratégia a ser implementada;
\item Implementação;
\item Testes/validação;
\item Escrita da Monografia; e
\item Defesa da Monografia.
\end{enumerate}

A Tab. \ref{tab:bullets}  é um exemplo típico de cronograma. O símbolo
‘\X’ foi usado neste exemplo, mas qualquer outro poderia ter sido
usado.

\newcommand{\X}{\textbullet}

 \begin{table}[ht]
 \centering
 \begin{tabular}{|c|c|c|c|c|c|}

  \hline
 &\multicolumn{2}{|c|}{\textbf{2016}}&\multicolumn{3}{|c|}{\textbf{2017}} \\
 \hline
 \hline
 Atividades & Nov. & Dez. & Jan. & Fev. & Mar. \\
 \hline
 1    &  \X  &  \X    &     &       &        \\ 
 2    &      &  \X    &     &       &        \\ 
 3    &      &  \X    &  \X &       &        \\ 
 4    &      &  \X    &  \X &  \X   &        \\
 5    &      &        &  \X &  \X   &        \\ 
 6    &      &        &     &       &  \X    \\ 
 \hline
 \hline 

 \end{tabular}
 \caption{Exemplo de cronograma usando \textit{bullets}}
 \label{tab:bullets}
 \end{table}




%%%%%%%%%%%%%%%%%%%%%%%%%%%%%%%%%%%%%%%
\singlespacing
\bibliographystyle{ufjfStyle}
\bibliography{referencias}

%%%%%%%%%%%%%%%%%%%%%%%%%%%%%%%%%%%%%%%%%%%%%%%%

\end{document}
