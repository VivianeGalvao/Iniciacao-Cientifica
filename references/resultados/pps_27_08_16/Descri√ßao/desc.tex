\documentclass[12pt]{article}

\usepackage[brazilian]{babel}
\usepackage[utf8]{inputenc}
\usepackage[T1]{fontenc}
\usepackage{graphicx}
\usepackage{geometry}
\usepackage{verbatim}
\usepackage[ruled, vlined]{algorithm2e}

\begin{document}

\section*{Métodos Implementados}
\subsection*{Evolução Diferencial}
A Evolução Diferencial (ED) é um método simples e eficiente, voltado para resolver problemas em domínio contínuo, guiado por mutação com vetor de diferenças. 

São usadas três variações desse método. Elas se diferenciam em relação ao tipo de mutação utilizado. Identificada como \textbf{ED/rand}, é a variação do ED que utiliza a regra de mutação em (\ref{eq::01}), a qual faz uma diferença ponderada entre dois indivíduos aleatórios $x_{i,r2}^t$ e  $x_{i,r3}^t$, para então perturbar o indivíduo $x_{i,r1}^t$ escolhido aleatoriamente, onde $r1\neq r2 \neq r3$. 

Outra variação é identificada como \textbf{ED/best}, que utiliza a mutação em (\ref{eq::02}), a qual realiza a diferença ponderada entre dois indivíduos aleatórios $x_{i,r1}^t$ e  $x_{i,r2}^t$, onde $r1\neq r2$, para fazer uma perturbação sobre melhor indivíduo encontrado.

A terceira variação do ED é identificada como \textbf{ED/target} e utiliza a mutação em (\ref{eq::03}), a qual usa duas diferenças ponderadas para perturbar o i-ésimo indivíduo da população $x_{i}^t$; uma diferença ponderada entre o melhor indivíduo encontrado e o i-ésimo indivíduo da população $x_{i}^t$ e outra diferença ponderada entre dois indivíduos aleatórios $x_{i,r1}^t$ e  $x_{i,r2}^t$, onde $r1\neq r2$. 
\begin{equation}
\label{eq::01}
	\nu_i^t = x_{i,r1}^t + F(x_{i,r2}^t - x_{i,r3}^t)
\end{equation} 
\begin{equation}
\label{eq::02}
	\nu_i^t = x_{best}^t + F(x_{i,r1}^t - x_{i,r2}^t)
\end{equation}
\begin{equation}
\label{eq::03}
	\nu_i^t = x_{i}^t + F(x_{best}^t - x_{i}^t) + F(x_{i,r1}^t - x_{i,r2}^t)
\end{equation} 

\subsection*{Evolução Diferencial com Probabilidade}
Nessa proposta de Evolução Diferencial, dado um tamanho $pop$ para a população, será construida uma população $POP = \alpha (pop)$, $\forall \alpha \in Z$. Assim, a cada geração do ED, $pop$ individuos serão escolhidos para participar da mutação, crossover e seleção. Essa escolha é feita através de um processo aleatório, onde cada indivíduo da população $POP$ tem $(1/\alpha) \%$ de serem escolhidos para a geração.
\subsection*{Evolução Diferencial com Torneio}
Nessa proposta de Evolução Diferencial, dado um tamanho $pop$ para a população, será construida uma população $POP = \alpha (pop)$, $\forall \alpha \in Z$. Assim, a cada geração do ED, $pop$  individuos serão escolhidos para participar da mutação, crossover e seleção. Essa escolha é feita através de um processo aleatório com um torneio, onde se o i-ésimo indivíduo da população for melhor que um indivíduo selecionado aleatoriomente, então ele é escolhido.
\section*{Resultados com Perfil de Desempenho}
São mostrados os gráficos dos resultados para cada variação do ED implementado com a proposta de aumento de diversidade na população. A intenção dos testes foi entender como é o comportamento do método quando se tem $\alpha \in \{2,3,4,5,6,7,8,9,10\}$. As legendas correspondem ao $\alpha$ usado no método, no modelo $N\alpha$. 
A legenda $N1$ refere-se ao método sem original, sem a adição de estratégia de seleção aleatória ou por torneio.

No gráfico `best x target' tem-se a comparação entre os métodos \textbf{ED/best} e \textbf{ED/target}, para $\alpha \in \{2,3\}$. As legendas deste gráfico têm significado diferente que as dos gráficos anteriores, onde $N1$ e $N2$ representam \textbf{ED/best} com Probabilidade, com $\alpha=2$ e $\alpha = 3$, respectivamente; $N3$ e $N4$ representam \textbf{ED/best} com Torneio, com $\alpha=2$ e $\alpha = 3$, respectivamente; $N5$ e $N6$ representam \textbf{ED/target} com Probabilidade, com $\alpha=2$ e $\alpha = 3$, respectivamente; $N7$ e $N8$ representam \textbf{ED/target} com Torneio, com $\alpha=2$ e $\alpha = 3$, respectivamente. 
\end{document}